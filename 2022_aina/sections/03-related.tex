\section{Related Work}
The data plane programmability has opened up a wide range of research opportunities to solve existing network monitoring problems such as packet priorization~\cite{tr19_p4_int_vnf}, the usage of selective telemetry to reduce the network overload~\cite{tr18_selective_in-band}, and the classification and analysis of network flows~\cite{tr19_flowstalker}. As previously mentioned, INT allows the network flow packets to embed network telemetry data - such as in \cite{tang2019sel,ben2020pint,tr21_lint}.
%
Sel-INT~\cite{tang2019sel} leverages select group tables to selectively insert INT headers into software switches based on its bucket's weight and a certain probability.
%
Similarly, PINT~\cite{ben2020pint} leverages global hash functions and randomly decides to embed INT data in a given packet,
%
while LINT~\cite{tr21_lint} implements a mechanism with adaptive telemetry accuracy, where each node in the network analyzes its impact and decides whether to send the information to the collector.

P4Entropy~\cite{tr20_estimating_log} calculates the entropy of traffic by using the information contained in the packets. The entropy results are forwarded to the controller for storage and future analysis. Lin~et~al.~\cite{tr21_network_telemetry_by} perform data collection to allow the SDN controller to evaluate the behavior of network traffic to improve data routing.
%
Likewise, \cite{tr20_multi-feature} proposes a DDoS schema entirely in the data plane. It leverages typical DDoS metrics such as incoming flows and packet symmetry ratio and periodically triggers alarms to external controllers.

With the emergency of hardware technologies such as SmartNICs and domain-specific programming languages such as P4, more applications are being implemented directly in the data plane. The implementation of selective and dynamic monitoring with the use of additional headers~\cite{tr19_sampling-based}, the classification of packets into classes with the implementation of machine learning algorithms~\cite{tr19_do_switches_dream} and the detection of flow events~\cite{tr20_flow_event} are some examples of programmatic packet header manipulation.
%
NIDS~\cite{tr20_anomaly_detection} consists of a machine learning-based anomaly detection algorithm for detecting anomalous packets and future error mitigation. %Likewise, \cite{tr20_detection_of_fog} analyzes and embeds telemetry data of interest to the applications.
SwitchTree \cite{tr20_switchtree} performs the implementation of the Random Forest algorithm in the data plane for packet analysis and decision-making while updating the rules at runtime. \cite{tr21_programmable_sw} and \cite{tr19_do_switches_dream, tr20_switchtree} present an implementation of machine learning models in the data plane using decision tree, and aim to generate network mapping for intrusion detection services.

Current research efforts in the INT domain (e.g., \cite{ben2020pint, tr21_lint, tang2019sel}) have mostly neglected the costs of reporting telemetry data to INT collectors. In fact, existing work have considered either a full report of information or a selective one based on static thresholds.
%
In this work, we aim to fill this gap and offload the decision process of reporting INT data into the SmartNIC data plane in a dynamically manner. We propose a lightweight in-network mechanism based on a window-based moving average with collected INT data as input.

%Current research efforts in the INT domain have mostly focused on using in-band telemetry as a way to solve network management and operations problems. implementing  mechanisms  to  to solve  only consider static telemetry reports to INT collectors.